\documentclass{article}

\usepackage{amssymb}
\usepackage{amsmath}

\usepackage{listings}

\usepackage{color}
\usepackage{xcolor}
\usepackage{graphicx}
\usepackage{fancyvrb}
\usepackage{multicol}

\usepackage{hyperref}

\usepackage{csvsimple}


\author{Evan Wilde}
\date{2016-11-14}
\title{Linear Programming Project}
\hypersetup{
  pdfauthor={Evan Wilde},
  pdftitle={Linear Programming Project},
}

% Evan Wilde
% Chart Colors
% This is the color scheme used by Google in the charts

\definecolor{chartblue}{HTML}{3366CC}
\definecolor{chartred}{HTML}{DC3912}
\definecolor{chartyellow}{HTML}{FF9900}
\definecolor{chartgreen}{HTML}{109618}
\definecolor{chartmagenta}{HTML}{990099}
\definecolor{chartpurple}{HTML}{3B3EAC}


\lstset{frame=tb,
  language=python,
  aboveskip=3mm,
  belowskip=3mm,
  showstringspaces=false,
  columns=flexible,
  basicstyle={\small\ttfamily},
  numbers=none,
  numberstyle=\tiny\color{gray},
  keywordstyle=\color{chartblue},
  commentstyle=\color{chartred},
  stringstyle=\color{chartgreen},
  breaklines=true,
  breakatwhitespace=true,
  tabsize=3
}

\begin{document}
\maketitle

\section{Basic Pedagogical Implementation}

\subsection{Input Files}

The program requires 3 files containing the A constraint coefficient matrix, the
boundary constraints (b vector), and the coefficients to the optimization function
(c vector).

The files representing each must be in the following order:
\begin{enumerate}
  \item A matrix
  \item b vector
  \item c vector
\end{enumerate}

The files containing the b and c vector do not require that the vector be written as
a column or row vector, and will allow the user to write them either way.

These two will behave equivalently:

\begin{multicols}{2}
\begin{minipage}[t]{0.5\textwidth}
\begin{Verbatim}
  1,2,3
\end{Verbatim}
\end{minipage}

\begin{minipage}[t]{0.5\textwidth}
  \begin{Verbatim}
  1
  2
  3
\end{Verbatim}
\end{minipage}
\end{multicols}

\textbf{Example of input files}

\begin{tabular}{c c c}
\begin{minipage}[t]{0.333\textwidth}
\textit{A.csv}
\VerbatimInput[frame=single]{examples/A.csv}
\end{minipage}

&

\begin{minipage}[t]{0.333\textwidth}
\textit{b.csv}
\VerbatimInput[frame=single]{examples/b.csv}
\end{minipage}

&

\begin{minipage}[t]{0.333\textwidth}
\textit{c.csv}
\VerbatimInput[frame=single]{examples/c.csv}
\end{minipage}
\end{tabular}

\subsection{Output}

%%TODO: Get the output

\subsection{Code}

\begin{lstlisting}

\end{lstlisting}

\end{document}
