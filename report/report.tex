\documentclass{article}

\usepackage{amssymb}
\usepackage{amsmath}

\usepackage{listings}

\usepackage{color}
\usepackage{xcolor}
\usepackage{graphicx}
\usepackage{fancyvrb}
\usepackage{multicol}

\usepackage{hyperref}

\usepackage{csvsimple}


\author{Evan Wilde}
\date{2016-11-14}
\title{Linear Programming Project}
\hypersetup{
  pdfauthor={Evan Wilde},
  pdftitle={Linear Programming Project},
}

% Evan Wilde
% Chart Colors
% This is the color scheme used by Google in the charts

\definecolor{chartblue}{HTML}{3366CC}
\definecolor{chartred}{HTML}{DC3912}
\definecolor{chartyellow}{HTML}{FF9900}
\definecolor{chartgreen}{HTML}{109618}
\definecolor{chartmagenta}{HTML}{990099}
\definecolor{chartpurple}{HTML}{3B3EAC}


\lstset{frame=tb,
  language=python,
  aboveskip=3mm,
  belowskip=3mm,
  showstringspaces=false,
  columns=flexible,
  basicstyle={\small\ttfamily},
  numbers=none,
  numberstyle=\tiny\color{gray},
  keywordstyle=\color{chartblue},
  commentstyle=\color{chartred},
  stringstyle=\color{chartgreen},
  breaklines=true,
  breakatwhitespace=true,
  tabsize=3
}

\begin{document}
\maketitle

\section{Basic Pedagogical Implementation}

\subsection{Input Files}

The program requires 3 files containing the A constraint coefficient matrix, the
boundary constraints (b vector), and the coefficients to the optimization function
(c vector).

The files representing each must be in the following order:
\begin{enumerate}
  \item A matrix
  \item b vector
  \item c vector
\end{enumerate}

The files containing the b and c vector do not require that the vector be written as
a column or row vector, and will allow the user to write them either way.

These two will behave equivalently:

\begin{multicols}{2}
\begin{minipage}[t]{0.5\textwidth}
\begin{Verbatim}
  1,2,3
\end{Verbatim}
\end{minipage}

\begin{minipage}[t]{0.5\textwidth}
  \begin{Verbatim}
  1
  2
  3
\end{Verbatim}
\end{minipage}
\end{multicols}

\textbf{Example of input files}

\begin{tabular}{c c c}
\begin{minipage}[t]{0.333\textwidth}
\textit{A.csv}
\VerbatimInput[frame=single]{examples/A.csv}
\end{minipage}

&

\begin{minipage}[t]{0.333\textwidth}
\textit{b.csv}
\VerbatimInput[frame=single]{examples/b.csv}
\end{minipage}

&

\begin{minipage}[t]{0.333\textwidth}
\textit{c.csv}
\VerbatimInput[frame=single]{examples/c.csv}
\end{minipage}
\end{tabular}

\subsection{Output}

\textbf{Iteration 1: }

\begin{center}
\begin{tabular}{cc}$B$: ${\left\{\begin{array}{ccc}3 & 4 & 5\end{array}\right\}}$ & $N$: ${\left\{\begin{array}{cc}1 & 2\end{array}\right\}}$\end{tabular}\\

\begin{tabular}{cc}$B$: $\left[\begin{array}{ccc}1.0 & 0.0 & 0.0\\ 0.0 & 1.0 & 0.0\\ 0.0 & 0.0 & 1.0\\ \end{array}\right]$ & $N$: $\left[\begin{array}{cc}1.0 & -1.0\\ 2.0 & -1.0\\ 0.0 & 1.0\\ \end{array}\right]$\end{tabular}\\

\begin{tabular}{cc}$x_B$: $\left[\begin{array}{c}1.0\\ 3.0\\ 5.0\\ \end{array}\right]$ & $z_N$: $\left[\begin{array}{c}-4.0\\ -3.0\\ \end{array}\right]$\end{tabular}\\

\end{center}
\textit{Step 1:}
Negative Coefficient(s) in $z_n$
\\\textit{Step 2:}\\
\begin{center}
$j$ = 1
\end{center}
\textit{Step 3:}\\
\begin{center}$\Delta x_B$ = $\left[\begin{array}{ccc}1.0 & 0.0 & 0.0\\ 0.0 & 1.0 & 0.0\\ 0.0 & 0.0 & 1.0\\ \end{array}\right]\left[\begin{array}{cc}1.0 & -1.0\\ 2.0 & -1.0\\ 0.0 & 1.0\\ \end{array}\right]\left[\begin{array}{c}1.0\\ 0.0\\ \end{array}\right] = \left[\begin{array}{c}1.0\\ 2.0\\ 0.0\\ \end{array}\right]$\end{center}
\textit{Step 4:}\\
\begin{center}
$t$ = 1.0
\end{center}
\textit{Step 5:}\\
\begin{center}
$i$ = 3
\end{center}
\textit{Step 6:}\\
\begin{center}$\Delta z_N = \left[\begin{array}{c}-1.0\\ 1.0\\ \end{array}\right]$\end{center}
\textit{Step 7:}\\
\begin{center}$s = 4.0$\end{center}
\textit{Step 8:}\\
\begin{center}
\begin{tabular}{cc}$x_1^* = 1.0$ & $x_B^* = \left[\begin{array}{c}0.0\\ 1.0\\ 5.0\\ \end{array}\right]$\end{tabular}
\\
\begin{tabular}{cc}$z_1^* = 4.0$ & $z_N^* = \left[\begin{array}{c}0.0\\ -7.0\\ \end{array}\right]$\end{tabular}
\end{center}
\textbf{Iteration 2: }

\begin{center}
\begin{tabular}{cc}$B$: ${\left\{\begin{array}{ccc}1 & 4 & 5\end{array}\right\}}$ & $N$: ${\left\{\begin{array}{cc}3 & 2\end{array}\right\}}$\end{tabular}\\

\begin{tabular}{cc}$B$: $\left[\begin{array}{ccc}1.0 & 0.0 & 0.0\\ 2.0 & 1.0 & 0.0\\ 0.0 & 0.0 & 1.0\\ \end{array}\right]$ & $N$: $\left[\begin{array}{cc}1.0 & -1.0\\ 0.0 & -1.0\\ 0.0 & 1.0\\ \end{array}\right]$\end{tabular}\\

\begin{tabular}{cc}$x_B$: $\left[\begin{array}{c}1.0\\ 1.0\\ 5.0\\ \end{array}\right]$ & $z_N$: $\left[\begin{array}{c}4.0\\ -7.0\\ \end{array}\right]$\end{tabular}\\

\end{center}
\textit{Step 1:}
Negative Coefficient(s) in $z_n$
\\\textit{Step 2:}\\
\begin{center}
$j$ = 2
\end{center}
\textit{Step 3:}\\
\begin{center}$\Delta x_B$ = $\left[\begin{array}{ccc}1.0 & 0.0 & 0.0\\ -2.0 & 1.0 & 0.0\\ 0.0 & 0.0 & 1.0\\ \end{array}\right]\left[\begin{array}{cc}1.0 & -1.0\\ 0.0 & -1.0\\ 0.0 & 1.0\\ \end{array}\right]\left[\begin{array}{c}0.0\\ 1.0\\ \end{array}\right] = \left[\begin{array}{c}-1.0\\ 1.0\\ 1.0\\ \end{array}\right]$\end{center}
\textit{Step 4:}\\
\begin{center}
$t$ = 1.0
\end{center}
\textit{Step 5:}\\
\begin{center}
$i$ = 4
\end{center}
\textit{Step 6:}\\
\begin{center}$\Delta z_N = \left[\begin{array}{c}2.0\\ -1.0\\ \end{array}\right]$\end{center}
\textit{Step 7:}\\
\begin{center}$s = 7.0$\end{center}
\textit{Step 8:}\\
\begin{center}
\begin{tabular}{cc}$x_2^* = 1.0$ & $x_B^* = \left[\begin{array}{c}2.0\\ 0.0\\ 4.0\\ \end{array}\right]$\end{tabular}
\\
\begin{tabular}{cc}$z_2^* = 7.0$ & $z_N^* = \left[\begin{array}{c}-10.0\\ 0.0\\ \end{array}\right]$\end{tabular}
\end{center}
\textbf{Iteration 3: }

\begin{center}
\begin{tabular}{cc}$B$: ${\left\{\begin{array}{ccc}1 & 2 & 5\end{array}\right\}}$ & $N$: ${\left\{\begin{array}{cc}3 & 4\end{array}\right\}}$\end{tabular}\\

\begin{tabular}{cc}$B$: $\left[\begin{array}{ccc}1.0 & -1.0 & 0.0\\ 2.0 & -1.0 & 0.0\\ 0.0 & 1.0 & 1.0\\ \end{array}\right]$ & $N$: $\left[\begin{array}{cc}1.0 & 0.0\\ 0.0 & 1.0\\ 0.0 & 0.0\\ \end{array}\right]$\end{tabular}\\

\begin{tabular}{cc}$x_B$: $\left[\begin{array}{c}2.0\\ 1.0\\ 4.0\\ \end{array}\right]$ & $z_N$: $\left[\begin{array}{c}-10.0\\ 7.0\\ \end{array}\right]$\end{tabular}\\

\end{center}
\textit{Step 1:}
Negative Coefficient(s) in $z_n$
\\\textit{Step 2:}\\
\begin{center}
$j$ = 3
\end{center}
\textit{Step 3:}\\
\begin{center}$\Delta x_B$ = $\left[\begin{array}{ccc}-1.0 & 1.0 & 0.0\\ -2.0 & 1.0 & 0.0\\ 2.0 & -1.0 & 1.0\\ \end{array}\right]\left[\begin{array}{cc}1.0 & 0.0\\ 0.0 & 1.0\\ 0.0 & 0.0\\ \end{array}\right]\left[\begin{array}{c}1.0\\ 0.0\\ \end{array}\right] = \left[\begin{array}{c}-1.0\\ -2.0\\ 2.0\\ \end{array}\right]$\end{center}
\textit{Step 4:}\\
\begin{center}
$t$ = 2.0
\end{center}
\textit{Step 5:}\\
\begin{center}
$i$ = 5
\end{center}
\textit{Step 6:}\\
\begin{center}$\Delta z_N = \left[\begin{array}{c}-2.0\\ 1.0\\ \end{array}\right]$\end{center}
\textit{Step 7:}\\
\begin{center}$s = 5.0$\end{center}
\textit{Step 8:}\\
\begin{center}
\begin{tabular}{cc}$x_3^* = 2.0$ & $x_B^* = \left[\begin{array}{c}4.0\\ 5.0\\ 0.0\\ \end{array}\right]$\end{tabular}
\\
\begin{tabular}{cc}$z_1^* = 5.0$ & $z_N^* = \left[\begin{array}{c}0.0\\ 2.0\\ \end{array}\right]$\end{tabular}
\end{center}
\textbf{Iteration 4: }

\begin{center}
\begin{tabular}{cc}$B$: ${\left\{\begin{array}{ccc}1 & 2 & 3\end{array}\right\}}$ & $N$: ${\left\{\begin{array}{cc}5 & 4\end{array}\right\}}$\end{tabular}\\

\begin{tabular}{cc}$B$: $\left[\begin{array}{ccc}1.0 & -1.0 & 1.0\\ 2.0 & -1.0 & 0.0\\ 0.0 & 1.0 & 0.0\\ \end{array}\right]$ & $N$: $\left[\begin{array}{cc}0.0 & 0.0\\ 0.0 & 1.0\\ 1.0 & 0.0\\ \end{array}\right]$\end{tabular}\\

\begin{tabular}{cc}$x_B$: $\left[\begin{array}{c}4.0\\ 5.0\\ 2.0\\ \end{array}\right]$ & $z_N$: $\left[\begin{array}{c}5.0\\ 2.0\\ \end{array}\right]$\end{tabular}\\

\end{center}
\textit{Step 1:}
Coefficients of $z_n$ are all positive


\subsection{Code}

\begin{lstlisting}

def solveLinearProgram(c, A, b):
    """solveLinearProgram

    Solve (maximizes) the linear program
    :c: Coefficients for the optimization function
    :A: Constraint Coefficient matrix
    :b: Constraint boundaries
    :returns: xb, zn, assignment, and solution to the linear program

    """
    if type(c) is not np.matrix:
        c = np.matrix(c)

    if type(A) is not np.matrix:
        A = np.matrix(A)

    if type(b) is not np.matrix:
        b = np.matrix(b)

    N = [i for i in range(A.shape[1])]
    B = [i + len(N) for i in range(A.shape[0])]

    zn = -c
    xb = b

    originalBasic = B[:]
    originalNonBasic = N[:]
    A = np.append(A, np.identity(len(B)), 1)

    def getBasicMatrix():
        """ Generate the basic matrix of the dictionary

        :returns: The basic matrix
        """
        return np.concatenate([A[:, i] for i in B], 1)

    def getNonBasicMatrix():
        """Generates the non-basic matrix

        :returns: The nonbasic matrix

        """
        return np.concatenate([A[:, i] for i in N], 1)

    while True:
        # Step 1/2
        firstPivot = minCoeff(zn)



        if firstPivot[0] == -1:
            var_assign = {idx : xb[B.index(n)].tolist()[0][0] if n in B else zn.transpose()[N.index(n)].tolist()[0][0] for idx, n in enumerate(originalNonBasic)}
            slack = {idx + len(originalNonBasic): zn.transpose()[N.index(n)].tolist()[0][0] if n in N else xb[B.index(n)].tolist()[0][0]  for idx, n in enumerate(originalBasic)}
            solution = sum([var_assign[index] * x for index, x in enumerate(np.nditer(c))])

            return {'xb': list(zip(B, [x for r in xb.tolist() for x in r])),
                    'zn': list(zip(N, [x for r in zn.tolist() for x in r])),
                    'originalVars': originalNonBasic,
                    'assignment': var_assign,
                    'slack': slack,
                    'solution': solution
                    }

        enteringIndex, value = firstPivot
        j = N[enteringIndex]

        # Step 3
        elementary = np.matrix(basisVector(getNonBasicMatrix().shape[1],
            enteringIndex)).transpose()
        binv = np.linalg.inv(getBasicMatrix())
        deltaxb = binv * getNonBasicMatrix() * elementary

        # Step 4
        numerators = [num.item(0) for num in np.nditer(deltaxb)]
        denominators = [num.item(0) if num.item(0) is not 0 else None for num in np.nditer(xb)]

        nums = [numerators[idx] / val for idx, val in enumerate(denominators) if val is not None]
        vectorIndex, t = max(enumerate(nums), key=operator.itemgetter(1))
        t = 1 / t

        # Step 5
        leavingIndex = B[vectorIndex]

        # Step 6
        elementary = np.matrix(basisVector(getNonBasicMatrix().shape[0],
            vectorIndex)).transpose()
        binvNT = (binv * getNonBasicMatrix()).transpose()
        deltazn = -binvNT * elementary

        # Step 7
        s = zn.item(enteringIndex) / deltazn.item(enteringIndex)

        # Step 8
        xb = xb - t * deltaxb
        zn = zn.transpose() - s * deltazn

        leavingIndexIndex = B.index(leavingIndex)
        enteringIndexIndex = N.index(j)

        B[leavingIndexIndex] = j
        N[enteringIndexIndex] = leavingIndex

        xb.put(leavingIndexIndex, t)
        zn.put(enteringIndexIndex, s)
        zn = zn.transpose()


\end{lstlisting}

\section{Degeneracies}

To work with the degenerate case of having a negative in the constraint boundaries,
we simply run the solver on the dual of the problem by taking the negative transpose
of the $A$ matrix, $c$ vector, and $b$ vector. Then feed that into the solver with
the original $c$ vector being fed in as the $b$ vector and the original $b$ vector
being fed in as the $c$ vector. The assignments in the slack output are the
assignments to the non-slack variables of the original problem.

\section{LaTeX Generation}

The code is updated to have a "debug" flag provided as a parameter enabling the
output of \LaTeX arrays.

The Simplex function is as follows

\begin{lstlisting}
def solveLinearProgram(c, A, b, debug=False):
    """solveLinearProgram

    Solve (maximizes) the linear program
    :c: Coefficients for the optimization function
    :A: Constraint Coefficient matrix
    :b: Constraint boundaries
    :returns: xb, zn, assignment, and solution to the linear program

    """
    if type(c) is not np.matrix:
        c = np.matrix(c)

    if type(A) is not np.matrix:
        A = np.matrix(A)

    if type(b) is not np.matrix:
        b = np.matrix(b)

    N = [i for i in range(A.shape[1])]
    B = [i + len(N) for i in range(A.shape[0])]

    zn = -c
    xb = b

    originalBasic = B[:]
    originalNonBasic = N[:]
    A = np.append(A, np.identity(len(B)), 1)

    def getBasicMatrix():
        """ Generate the basic matrix of the dictionary

        :returns: The basic matrix
        """
        return np.concatenate([A[:, i] for i in B], 1)

    def getNonBasicMatrix():
        """Generates the non-basic matrix

        :returns: The nonbasic matrix

        """
        return np.concatenate([A[:, i] for i in N], 1)

    if debug:
        iterations = 1
    while True:

        if debug:
            # Print basic info
            print(r"\textbf{Iteration " + str(iterations) + r": }" + "\n")
            BIndex = r"$\mathcal{B} = " + \
                    r"\left\{\begin{array}{" + "c" * len(B) + r"}" +\
                    " & ".join([str(x + 1) for x in B]) + \
                    r"\end{array}\right\}$"
            NIndex = r"$\mathcal{N} = " + \
                    r"\left\{\begin{array}{" + "c" * len(N) + r"}" +\
                    " & ".join([str(x + 1) for x in N]) + \
                    r"\end{array}\right\}$"
            print(LaTeX.LaTeXCenter([
                LaTeX.LaTeXSingleLine([BIndex, NIndex]), # The indices
                LaTeX.LaTeXSingleLine([
                    "$N = " + LaTeX.LaTeXMatrixToArray(getBasicMatrix()) + "$",
                    "$B = " + LaTeX.LaTeXMatrixToArray(getNonBasicMatrix()) + "$"
                    ]),  # The matrices
                LaTeX.LaTeXSingleLine([ "$x_B^* = " + LaTeX.LaTeXMatrixToArray(xb) + "$",
                    "$z_n^* = " + LaTeX.LaTeXMatrixToArray(zn.transpose()) + "$"
                    ]) # xb* and zn* vectors
                ]))

        if debug:
            print(r"\textit{Step 1:}")

        # Step 1/2
        firstPivot = minCoeff(zn)
        if firstPivot[0] == -1:
            var_assign = {idx : xb[B.index(n)].tolist()[0][0] if n in B else zn[N.index(n)].tolist()[0][0] for idx, n in enumerate(originalNonBasic)}
            solution = sum([var_assign[index] * x for index, x in enumerate(np.nditer(c))])

            if debug:
                print("$z_n$ does not contain any negative values. Solution: {}".format(solution))

            return {'xb': list(zip(B, [x for r in xb.tolist() for x in r])),
                    'zn': list(zip(N, [x for r in zn.tolist() for x in r])),
                    'originalVars': originalNonBasic,
                    'assignment': var_assign,
                    'solution': solution
                    }

        

        enteringIndex, value = firstPivot
        j = N[enteringIndex]

        if debug:
            print("$z_n$ contains negative values. Continue optimizing.\n")
            print(r"\textit{Step 2:}\\")
            print(LaTeX.LaTeXCenter(['$j = {}$'.format(j + 1)]))

        # Step 3
        elementary = np.matrix(basisVector(getNonBasicMatrix().shape[1],
            enteringIndex)).transpose()
        binv = np.linalg.inv(getBasicMatrix())
        deltaxb = binv * getNonBasicMatrix() * elementary

        if debug:
            print(r"\textit{Step 3:}\\")
            print("$$\Delta x_b = {} {} = {}$$".format(
                LaTeX.LaTeXMatrixToArray(binv),
                LaTeX.LaTeXMatrixToArray(elementary),
                LaTeX.LaTeXMatrixToArray(deltaxb)))

        # Step 4
        numerators = [num.item(0) for num in np.nditer(deltaxb)]
        denominators = [num.item(0) if num.item(0) is not 0 else None for num in np.nditer(xb)]

        nums = [numerators[idx] / val for idx, val in enumerate(denominators) if val is not None]
        vectorIndex, t = max(enumerate(nums), key=operator.itemgetter(1))
        t = 1 / t

        if debug:
            print(r"\textit{Step 4:}\\")
            print("$$t = {}$$".format(t))

        # Step 5
        leavingIndex = B[vectorIndex]

        if debug:
            print(r"\textit{Step 5:}\\")
            print("$$i = {}$$".format(leavingIndex + 1))

        # Step 6
        elementary = np.matrix(basisVector(getNonBasicMatrix().shape[0],
            vectorIndex)).transpose()
        binvNT = (binv * getNonBasicMatrix()).transpose()
        deltazn = -binvNT * elementary

        if debug:
            print(r"\textit{Step 6:}\\")
            print("$$\Delta z_n = {} {} = {}$$".format(
                LaTeX.LaTeXMatrixToArray(binvNT),
                LaTeX.LaTeXMatrixToArray(elementary),
                LaTeX.LaTeXMatrixToArray(deltazn)))

        # Step 7
        s = zn.item(enteringIndex) / deltazn.item(enteringIndex)

        if debug:
            print(r"\textit{Step 7:}\\")
            print("$$s = {}$$".format(s))

        # Step 8
        xb = xb - t * deltaxb
        zn = zn.transpose() - s * deltazn

        if debug:
            oldXB = LaTeX.LaTeXMatrixToArray(xb)
            oldZN = LaTeX.LaTeXMatrixToArray(zn)

        leavingIndexIndex = B.index(leavingIndex)
        enteringIndexIndex = N.index(j)

        B[leavingIndexIndex] = j
        N[enteringIndexIndex] = leavingIndex

        xb.put(leavingIndexIndex, t)
        zn.put(enteringIndexIndex, s)

        if debug:
            print(r"\textit{Step 8:}\\")

            print(LaTeX.LaTeXCenter([
                LaTeX.LaTeXSingleLine([
                    "$x_{}^* = {}$".format(leavingIndex + 1, t),
                    "$x_B^* = " + oldXB + "$"
                    ]),
                LaTeX.LaTeXSingleLine([
                    "$z_{}^* = {}$".format(enteringIndexIndex + 1, s),
                    "$z_N^* = " + oldZN + "$"
                    ])
                ]))
            iterations += 1

        zn = zn.transpose()
\end{lstlisting}

We include an additional library of \LaTeX generators that enable us to quickly
create multiple items on a single line (using a tabular environment) and center
using the center environment. Furthermore, it is able to generate matrices using
array environments in the math mode.

The code in the library is as follows

\begin{lstlisting}
#
# Utilities to help when working with LaTeX in Python
# Evan Wilde
#

import numpy as np

def LaTeXMatrixToArray(matrix):
    """Prints the matrix in LaTeX form

    :matrix: numpy matrix
    :returns: String of the matrix

    """
    dims = matrix.shape
    # print(dims)  # dims[0]: height, dims[1]: width
    opening = r"\left[\begin{array}{" + "c" * dims[1] + r"}"
    closing = r"\end{array}\right]"
    string = ""
    for idx, x in enumerate(np.nditer(matrix)):
        string += str(float(x))
        if idx % dims[1] == dims[1] - 1:
            string += r"\\ "
        else:
            string += " & "
    return opening + string + closing

def LaTeXSingleLine(lst):
    """Puts strings from list into an nx1 tabular

    :lst: The list of strings to be put into the tabular
    :returns: stirng of tabular environment
    """
    start = r"\begin{tabular}{" + "c" * len(lst) + r"}"
    inner = " & ".join(lst)
    end = r"\end{tabular}"
    return start + inner + end

def LaTeXCenter(lst):
    """Prints contents of list in center environment

    Each element of the list is put on a separate line (should be LaTeX
    formatted strings)
    :returns: string with stuff in the center
    """
    start = r"\begin{center}" + '\n'
    inner = '    ' + r"\\ ".join(lst) + r"\\"
    end   = '\n' + r"\end{center}" + '\n'
    return start + inner + end
\end{lstlisting}

\section{Criss-Cross}

I have no idea...

\end{document}
